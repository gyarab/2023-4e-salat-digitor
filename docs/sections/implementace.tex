\section{Implementace v programu}
Program slouží k vytváření a trénování neuronových sítí podle vstupních parametrů.
Samotný program se dá spustit s přepínači, které určují chování programu.
Program je schopen na požadavek buď, vytvořit novou síť podle vstupních parametrů, nebo načíst existující síť ze souboru JSON.
S takto vytvořenou neuronovou sítí je program dále schopen pracovat dvěma způsoby.
Neuronovou síť může trénovat nebo jí jen načte a čeká na vstupní data, která neuronová síť zpracuje a vrátí výsledek.

\subsection{Vytváření a načítání neuronové sítě}
Neuronová síť je v programu reprezentována vícerozměrnými poli. První dvojrozměrné pole reprezentuje neurony.

Pole `neuron` má dva rozměry tzn. pole v poli.
První (nulté) pole v `neuron` reprezentuje první vrstvu neuronů (vstupní) vrstvu.
Naopak poslední prvek pole `neuron` reprezentuje poslední (výstupní) vrstvu.
Počet skrytých vrstev v neuronové síti se tedy rovná `neuron.size()-2`.
Každé toto pole uložené v `neuron` obsahuje `n` hodnot, které reprezentují hodnoty konkrétních neuronů. Hodnoty jsou typu long double.

Další pole `bias` má také dva rozměry a jeho velikost je identická jako velikost pole `neuron`.
Hodnoty v jednotlivých polích reprezentují hodnotu biasu konkrétního neuronu. Hodnoty jsou také typu long double.

Poslední pole reprezentující samotnou neuronovou síť je trojrozměrné pole `weight`. Velikost pole je rovno počtu vrstev mínus jedna.
První (nultý) prvek tj. `weight[0]` reprezentuje dvojrozměrné pole, které reprezentuje váhy spojující vstupní neuroný s neurony druhé vrstvy.
Pole `weight[0][0]` je pole hodnot jednotlivých vah, které míří do prvních (nultého) neuronu druhé vrstvy. Váhy jsou také typu long double.

Při vytváření nové neuronové sítě se v konstruktoru vytvoří zmiňovaná pole o velikostech, které odpovídají zadaným parametrům.
Hodnoty vah a biasů se incializují s náhodnou hodnotou. Dále se celá neuronová síť serializuje do souboru JSON, který se následně uloží.
V názvu souboru je obsažen počet vrstev sítě navíc s náhodným číslem, které se pokusí zabránit kolizi na disku. Soubor obsahuje všechny potrebné data pro pozdější načtení neuronové sítě.
To znamená velikost, aktivační funkce neuronů, hodnoty vah a hodnoty biasů.

Pokud program pouze načítá neuronovou síť z již existujícího souboru, konstruktor pouze očekává název souboru.
Dále konstruktor deserializuje data ze souboru a uloží si je do proměnných.

\subsection{Trénování neuronové sítě}

\subsection{Spouštění programu}
