\section{Implementace v programu}
Program slouží k vytváření a trénování neuronových sítí podle vstupních parametrů.
Samotný program se dá spustit s přepínači, které určují chování programu.
Program je schopen na požadavek buď vytvořit novou síť podle vstupních parametrů, nebo načíst existující síť ze souboru JSON.
S takto vytvořenou neuronovou sítí je program dále schopen pracovat dvěma způsoby.
Neuronovou síť může trénovat nebo jí jen načte a čeká na vstupní data, která neuronová síť zpracuje a vrátí výsledek.

\subsection{Vytváření a načítání neuronové sítě}
Neuronová síť je v programu reprezentována vícerozměrnými poli. První dvojrozměrné pole reprezentuje neurony.

Pole \(neuron\) má dva rozměry tzn. pole v poli.
První (nulté) pole v \(neuron\) reprezentuje první vrstvu neuronů (vstupní) vrstvu.
Naopak poslední prvek pole \(neuron\) reprezentuje poslední (výstupní) vrstvu.
Počet skrytých vrstev v neuronové síti se tedy rovná \((neuron.size() - 2)\).
Každé toto pole uložené v \(neuron\) obsahuje \(n\) hodnot, které reprezentují hodnoty konkrétních neuronů. Hodnoty jsou typu long double.

Dále je potřeba pole \(rawNeuron\), které je uplně stejné jako pole \(neuron\), ale jsou do něj ukládány neaktivované hodnoty neuronů.

Další pole \(bias\) má také dva rozměry a jeho velikost je identická jako velikost pole \(neuron\).
Hodnoty v jednotlivých polích reprezentují hodnotu biasu konkrétního neuronu. Hodnoty jsou také typu long double.

Poslední pole reprezentující samotnou neuronovou síť je trojrozměrné pole \(weight\). Velikost pole je rovno počtu vrstev mínus jedna.
První (nultý) prvek tj. \(weight[0]\) reprezentuje dvojrozměrné pole, které reprezentuje váhy spojující vstupní neurony s neurony druhé vrstvy.
Pole \(weight[0][0]\) je pole hodnot jednotlivých vah, které míří do prvních (nultého) neuronu druhé vrstvy. Váhy jsou také typu long double.

Při vytváření nové neuronové sítě se v konstruktoru vytvoří zmiňovaná pole o velikostech, které odpovídají zadaným parametrům.
Hodnoty vah a biasů se incializují s náhodnou hodnotou. Dále se celá neuronová síť serializuje do souboru JSON, který se následně uloží.
V názvu souboru je obsažen počet vrstev sítě navíc s náhodným číslem, které se pokusí zabránit kolizi na disku.
Soubor obsahuje všechny potrebné data pro pozdější načtení neuronové sítě.
To znamená velikost, aktivační funkce neuronů, hodnoty vah a hodnoty biasů.

Pokud program pouze načítá neuronovou síť z již existujícího souboru, konstruktor pouze očekává název souboru.
Dále konstruktor deserializuje data ze souboru a uloží si je do proměnných.

\subsection{Trénování neuronové sítě}
Jak už bylo zmíněno v kapitole o strojovém učení, proces učení neuronové sítě se skládá ze čtyř částí. Tedy průchod dat sítí a vyhodnocení chyby,
zpětné počítání chyby, aktualizace vah a biasů a nakonec se vše zopakuje. Průchod dat sítí neboli přímý průchod je vyřešen jako tři vnořené \(for\) cykly.
Hodnota každého neuronu se vypočítá podle vzorce na výpočet hodnoty neuronu, a tudíž \(z_{j}^{l} = \left( \sum (w^{l}_{jk} \cdot a^{l-1}_k) + b^l_j \right)\).
Z důvodu, že při zpětném počítání chybe je potřeba aktivovaný i neaktivovaný neuron, tak se do paměti uloží obě hodnoty (do pole \(neuron\) a \(rawNeuron\)).
Pro samotné učení je v programu metoda \(train\), která všechny tyto kroky provede.

Při volání metody \(train\), funkce očekává dvojrozměrné pole typy \(TrainData\), které obsahuje vstupní hodnoty a správný výsledek průchodu,
počet trénovacích cyklů (iterací) a jako poslední parametru očekává tzv. rychlost učení (learning rate).
Learning rate určuje míru změny při aktualizaci vah a biasů. Hodnota rychlosti učení se běžně pohubuje v rozmezí od 0.1 do 0.0001.
Pole \(TrainData\) je dvojrozměrné z důvodu, že data jsou rozdělena do tzv. seté, které zajišťují rychlejší a univerzálnější učení.
Tím pádem metoda \(train\) funguje tak, že zavolá metodu \(backpropagate\) na všechny členy setu.
Aktualizované hodnoty vah a bíasů se ukládají do externího pola a skutečné váhy a biasy neuronové sítě se aktualizují až po dokončení.
Takto se postupně trénuje síť pomocí všech setů dat. Následně se celý tento proces opakuje podle počtu iterací.
Metoda v průběhu počítá průměrnou chybu pro vstupní data, kterou vypisuje na standardní výstup společně s progresem trénování (procento vykonaných iterací).

Metoda \(backpropagate\) funguje na principu popsaném v kapitole \ref{strojove_uceni} strojové učení. Funkce je rozdělena do dvou částí.
První část spočíta chybu pro poslední vrstvu, která se počítá jednodušeji, protože neurony ovlivňují chybu pouze jednou cestou.
Druhá část naváže na první a počítá zpětně chybu pro zbytek vrstev. Pro výpočty jsou použity vzorce z kapitoly \ref{strojove_uceni} o strojovém učení.

\subsection{Spouštění programu}
Při spouštění si pomocí přepínačů lze nastavit, zda chceme například vytvářet úplně novou neuronovou síť, nebo chceme načíst již exístující síť. Dále je možné specifikovat jestli chceme síť trénovat nebo ji chceme použít k zpracování dat.

Možné použití jsou:
\begin{itemize}
    \item Načte neuronovou síť ze souboru a čeká na standartním vstupu data na zpracování.
    \begin{lstlisting}[language=bash, backgroundcolor=\color{backcolor}]
 $ ./digitor <jmeno_souboru>
    \end{lstlisting}

    \item Vytvoří neuronovou síť podle zadaných parametrů.
    Formát pro zadání neuronů je vždy počet neuronů pro každou vrstvu, který je oddělen čárkou.
    Př. "10,2,2,10", tato síť by měla čtyři vrstvy, z čehož vstupní i výstupní by měla 10 neuronů a obě skryté by měly 2 neurony.
    \begin{lstlisting}[language=bash, backgroundcolor=\color{backcolor}]
 $ ./digitor -n <neurony> <aktivace>
    \end{lstlisting}

    \item Trénování exitující sítě. Program dostane informace o počtu učících dat, počtu iterací a míře učení.
    Dále program čeká na standartní vstup zmiňované učící data a následovně síť trénuje.
    \begin{lstlisting}[language=bash, backgroundcolor=\color{backcolor}]
 $ ./digitor -t <jmeno_soubor> <pocet_iteraci> <rychlost_uceni>
    <pocet_davek> <pocet_dat>
    \end{lstlisting}

    \item Vytvoří novou neuronovou síť podle požadavků a síť natrénuje stejným způsobem jako v předešlém bodě.
    \begin{lstlisting}[language=bash, backgroundcolor=\color{backcolor}]
 $ ./digitor -t -n <neurony> <iterace> <rychlost_uceni>
    <aktivace> <pocet_davek> <pocet_dat>
    \end{lstlisting}
\end{itemize}
