\section{Úvod}
Cílem maturitní práce je vytvořit neuronovou síť a naprogramovat k ní základní učící algoritmus bez použití knihoven pro strojové učení (jako TensorFlow, nebo PyTorch).
Záměrem je hlubší pochopení a prozkoumání různých implementací neuronových sítí.
Neuronová síť Digitor bude natrénována pro rozpoznávání rukou napsaných číslic [0-9].
Síť dostane na vstup čtvercový obrázek a výstupem bude číslice obsažena v obrázku.

\subsection{Výběr tématu}
Toto téma jsem si vybral z důvodu, že si myslím, že neuronové sítě jsou velice zajímavé a dosud ne úplně prozkoumané téma.
Věřil jsem, že tvorbou tohoto projektu se naučím jak neuronové sítě fungují.
Zárověn bych na toto téma rád navázal při studiu na vyskoké škole.

\subsection{Použité technologie}
Hlavní část tohoto projektu byla naprogramována v jazyce C++.
Jazyk jsem vybral z důvodu, že je velice rychlý a to za cenu, že uživatelský komfort je horší než například při použití Pythonu.
Jak už bylo zmíněno, pro tento projekt byla nejdůležitější rychlost, a proto jsem vybíral mezi jazykem C a C++.
Nakonec jsem zvolil C++ z důvodu, že i přes svoji rychlost a relativně velkou kontrolou je objektově orientován a obecně uživatelsky přívětivější.
Konkrétně byla použita verze C++23.

Dále byl použit nástroj CMake, který slouží k buildění C++ kódu.
CMake umožňuje zjednodušení celého buildovacího procesu a dokáže zajistit přenositelnost.

Pro práci s formátem JSON jsem zvolil C++ knihovnu `nlohmann/json`\cite{json}.

Při práci s obrázky byl použit jazyk Python. V tomto jazyce byly napsány různé scripty,
které dokáží načíst obrázky ze souboru a převést je do formátu, se kterým dokáže neuronová síť dále pracovat.

