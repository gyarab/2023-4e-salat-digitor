\section{Zkoumání architektur}
Při vytváření sítě na rozpoznávání rukou napsaných číslic jsme narazil na problém, že nevím jak velkou sít na to mám použít.
Při hledání na internetu jsem zjistil, že vlastě není žádná perfektní síť, a že ideální velikost sítě vždy závisí na problému, který má řešit.
Z toho důvodu jsem se rozhodl, že vytvořím malý výzkum na to, které síť má nejlepší výsledky.
Vytvořil jsem proto osm různých neuronových sítí, kterým jsem dal úplně stejný učící set a stejný počet iterací na natrénování.
Sítě měly vždy stejný počet vstupních a výstupních neuronů pouze se lišily v počtu skrytých vrstev a v počtu neuronů v těchto vrstvách.
Použité architektury byly následovné:
\begin{itemize}
    \item 2 vrstvy po 64 neuronech
    \item 3 vrstvy po 16 neuronech
    \item 3 vrstvy po 24 neuronech
    \item 3 vrstvy po 32 neuronech
    \item 3 vrstvy po 40 neuronech
    \item 3 vrstvy po 48 neuronech
    \item 3 vrstvy po 56 neuronech
    \item 3 vrstvy po 64 neuronech
\end{itemize}
Tyto konkrétní architektury byly zvoleny z důvodu, že se mi ještě před ukutečí tohoto zkoumání podařilo vytvořit model,
který měl 3 skryté vrstvy po 64 neuronech, a který dokázal rozpoznat obrázky s vysokou přesnotsí.
Proto mě napadlo zkusit kolik neuronů je potřeba, aby neuronová síť měla uspokujivé výsledky.
Zvolil jsme tedy neuronové sítě, které mají postupně nižsí počet neuronů a zkoumal jsem jaký to má vliv na výsledky.

Vytvořil jsem tedy 8 těchto sítí a spustiol jsme trénování. Všechný sítě dostaly identická data a identický počet iterací.
Zárověň jsem měřil čas, za jak dlouho se síť s těmito parametry natrénuje. Sítě byly tedy na trénovány pomocí 10000 obrázků (1000 z od každé číslice).
Po ukončení trénování jsem sítě otestoval pomocí 27730 vyčleněných obrázků z databáze mnist, které sítě nikdy neviděli.

\begin{center}
    \begin{tabular}{||c c c||}
        \hline
        Architektura        & Čas učení   & Přesnost \\ [0.5ex]
        \hline\hline
        2 vrstvy 64 neuronů & 20h 5m 31s  & 53.0\%   \\
        \hline
        3 vrstvy 16 neuronů & 5h 46m 27s  & 24.7\%   \\
        \hline
        3 vrstvy 24 neuronů & 8h 30m 19s  & 34.6\%   \\
        \hline
        3 vrstvy 32 neuronů & 10h 42m 14s & 54.5\%   \\
        \hline
        3 vrstvy 40 neuronů & 13h 49m 14s & 50.9\%   \\
        \hline
        3 vrstvy 48 neuronů & 16h 48m 6s  & 57.9\%   \\
        \hline
        3 vrstvy 56 neuronů & 19h 54m 12s & 55.2\%   \\
        \hline
        3 vrstvy 64 neuronů & 23h 25m 54s & 58.4\%   \\
        \hline
    \end{tabular}
\end{center}

Z výsledků je zřejmé, že sítě s vyšším počtem neuronů mají v tomto případě lepší výsledky, ale trénování takových sítí trvá delší čas.
V tomto připadě jsem došel k závěru, že rozdíl mezi trénováním menších a větších sítí není tak markantní,
a tudíž se v tomto konkrétním připadě vyplatí použít sítě s většim počtem vrstev a neuronů.
Největší síť se dokázala vytrénovat v pozdějším trénování vytrénovat natolik, že dokázala poznat více než 98\% obrázků, které nikdy neviděla.