\section{Závěr}
Při vytváření tohoto projektu jsem se ponořil do problematiky neuronových sítí a zjistil jsem, na jakém principu funguje strojové učení pomocí zpětného počítaní chyby.
Tento projekt mi pomohl vytvořit si představu, jak neuronové sítě fungují a k čemu vůbec mohou sloužit.

\subsection{Prostor pro zlepšení}
Při trénování modelu by se měla použít metoda špatných dat. To znamená, že se při trénování dávají modelu nejen správná data,
které se model má naučit, ale i chybná data, která má síť vyhodnodit jako chybná. Tím se zabrání tomu, že model vyhodnotí úplně špatná data jako správná.
To jsem při trénování hlavního modelu na rozpoznávání číslic nepoužil.
Zaprvé z důvodu, že jsem neměl přístup k vélkému objemu chybných dat, zadruhé proto, že o existenci této metody jsem se dozvěděl až po ukončení trénování modelu.
Při trénování dalších modelů bych tuto metodu určitě implementoval.

V budoucnu bych také chtěl v programu implementovat parelelní výpočet pomocí CUDA. Díky tomu by se mohlo trénování sítí řádově urychlit.
Dále bych chtěl v budoucnu přidat možnost pro vytváření jiných typů neuronových sítí než jen plně propojených.