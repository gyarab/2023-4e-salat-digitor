%%% -------------------- Titulní strana --------------------
\begin{titlepage}
    \begin{center}
        \large \vspace*{\fill}
        \thispagestyle{empty}

        \LARGE

        { \huge \textbf{Gymnázium Arabská, Praha 6, Arabská 14}}

        {\LARGE Obor programování }

        \vfill
        \includegraphics{logogyarab.png}
        \vspace{15pt}

        \vfill

        {\huge \textbf{Neuronová síť: Digitor}}

        \vfill

        Ondřej Salát

        \vfill

        {\large Duben, 2024}

        \vspace*{\fill}
    \end{center}
\end{titlepage}

%%% -------------------- Prohlášení --------------------
\thispagestyle{empty}
\addtocounter{page}{-1}
\vspace*{\fill}
Prohlašuji, že jsem jediným autorem tohoto projektu, všechny citace jsou řádně označené a všechna
použitá literatura a další zdroje jsou v práci uvedené.
Tímto dle zákona 121/2000 Sb. (tzv.\ Autorský zákon)
ve znění pozdějších předpisů uděluji bezúplatně škole Gymnázium, Praha 6, Arabská 14 oprávnění k výkonu
práva na rozmnožování díla (§ 13) a práva na sdělování díla veřejnosti (§ 18) na dobu časově neomezenou a
bez omezení územního rozsahu.

\vspace{2cm}
V .......... dne ............... \hspace{4cm} Ondřej Salát .................

\vspace{2cm}

%%% -------------------- Anotace --------------------
\newpage
\begin{abstract}
    Práce se zabývá procesem tvorby programu na vytváření a trénování neuronových sítí.
    Popisuje princip strojového učení za použití algoritmu backpropagation.
    Dále se práce věnuje pokusu o hlubší pochopení neuronových sítí.
    Konkrétně se věnuje různým architekturám neuronové sítě pro rozpoznávání rukou napsaných cifer.
\end{abstract}

%%% -------------------- Obsah --------------------
\tableofcontents